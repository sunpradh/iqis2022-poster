%% THEMING
%%--------------------------------------------------

%%--------------------------------------------------
%% FONTS
%%--------------------------------------------------
\usepackage{fontspec}
\usepackage{anyfontsize}
\usepackage[sfdefault]{FiraSans}
\usepackage[scaled=1.15]{newtxmath}

%&--------------------------------------------------
%& COLORS
%%--------------------------------------------------

%%% IBM Carbon colorscheme
%%% https://carbondesignsystem.com/guidelines/color/overview/
\definecolor{Foreground}{RGB}{22, 22, 22}
\definecolor{Background}{RGB}{244, 244, 244}
\definecolor{Blue100}{RGB}{0, 17, 65}
\definecolor{Blue90}{RGB}{0, 29, 108}
\definecolor{Blue80}{RGB}{0, 45, 156}
\definecolor{Blue70}{RGB}{0, 67, 206}
\definecolor{Blue60}{RGB}{15, 95, 254}
\definecolor{Blue50}{RGB}{69, 137, 255}
\definecolor{Blue40}{RGB}{120, 169, 255}
\definecolor{Blue30}{RGB}{166, 200, 255}
\definecolor{Blue20}{RGB}{208, 226, 255}
\definecolor{Blue10}{RGB}{237, 245, 255}
\definecolor{Red}{RGB}{218, 30, 40}
\definecolor{Orange}{RGB}{255, 131, 43}
\definecolor{Yellow}{RGB}{253, 220, 105}
\definecolor{Green}{RGB}{25, 128, 56}
\definecolor{Grey05}{RGB}{240, 240, 240}
\definecolor{Grey10}{RGB}{224, 224, 224}
\definecolor{Grey15}{RGB}{211, 211, 211}
\definecolor{Grey20}{RGB}{198, 198, 198}
\definecolor{Grey30}{RGB}{168, 168, 168}
\definecolor{Grey40}{RGB}{141, 141, 141}
\definecolor{Grey50}{RGB}{111, 111, 111}
\definecolor{Grey60}{RGB}{82, 82, 82}
\definecolor{Grey70}{RGB}{57, 57, 57}
\definecolor{Grey80}{RGB}{38, 38, 38}
\definecolor{Grey90}{RGB}{22, 22, 22}
\definecolor{White}{RGB}{255, 255, 255}
\definecolor{Black}{RGB}{0, 0, 0}

\definecolorstyle{MyColorStyle}{
    % Random colors
    \definecolor{colorOne}{named}{Blue70}
    \definecolor{colorTwo}{named}{Green}
    \definecolor{colorThree}{named}{Red}
}{
    % Background Colors
    \colorlet{backgroundcolor}{White}
    \colorlet{framecolor}{Grey50}
    % Title Colors
    \colorlet{titlefgcolor}{Background}
    \colorlet{titlebgcolor}{colorOne}
    % Block Colors
    \colorlet{blocktitlebgcolor}{Background}
    \colorlet{blocktitlefgcolor}{colorOne}
    \colorlet{blockbodybgcolor}{Background}
    \colorlet{blockbodyfgcolor}{Foreground}
    % Innerblock Colors
    \colorlet{innerblocktitlebgcolor}{Grey10}
    \colorlet{innerblocktitlefgcolor}{colorTwo}
    \colorlet{innerblockbodybgcolor}{Grey10}
    \colorlet{innerblockbodyfgcolor}{Foreground}
    % Note colors
    \colorlet{notefgcolor}{black}
    \colorlet{notebgcolor}{colorTwo!50!white}
    \colorlet{noteframecolor}{colorTwo}
}

\usecolorstyle{MyColorStyle}


%%--------------------------------------------------
%% BACKGROUND
%%--------------------------------------------------
\definebackgroundstyle{MyBackground}{
    \fill[
        backgroundcolor,
        inner sep=0pt,
        line width=0pt,
    ] (bottomleft) rectangle (topright);
}

\usebackgroundstyle{MyBackground}


%%--------------------------------------------------
%% BLOCKS
%%--------------------------------------------------
\defineblockstyle{MyBlockStyle}{
    titlewidthscale=1,
    bodywidthscale=1,
    titlecenter,
    titleoffsetx=0pt,
    titleoffsety=10pt,
    bodyoffsetx=0mm,
    bodyoffsety=15mm,
    bodyverticalshift=5mm,
    roundedcorners=0,
    linewidth=1pt,
    titleinnersep=1.2cm,
    bodyinnersep=1.2cm
}{
    \ifBlockHasTitle
        % Body
        \fill[
            blockbodybgcolor,
            rounded corners=\blockroundedcorners,
            line width=\blocklinewidth
        ] (blockbody.south west) rectangle (blockbody.north east);
        % Block Title
        \fill[
            fill=blockbodybgcolor,
            rounded corners=\blockroundedcorners,
            line width=\blocklinewidth
        ] (blocktitle.south west) rectangle (blocktitle.north east);
    \else
        % Only body
        \fill[
            blockbodybgcolor,
            rounded corners=\blockroundedcorners,
            line width=2\blocklinewidth
        ] (blockbody.south west) rectangle (blockbody.north east);
    \fi
}

\defineinnerblockstyle{MyInnerBlockStyle}{
    titlewidthscale=1,
    bodywidthscale=1,
    titleleft,
    titleoffsetx=0pt,
    titleoffsety=0pt,
    bodyoffsetx=0mm,
    bodyoffsety=0mm,
    bodyverticalshift=-10mm,
    roundedcorners=0,
    linewidth=1pt,
    titleinnersep=0.8cm,
    bodyinnersep=0.8cm
}{
    % Body
    \fill[
        innerblockbodybgcolor,
        rounded corners=0,
        line width=\innerblocklinewidth
    ] (innerblockbody.south west) rectangle (innerblockbody.north east);
    % Block Title
    \fill[
        innerblocktitlebgcolor,
        rounded corners=0,
        line width=\innerblocklinewidth
    ] (innerblocktitle.south west) rectangle (innerblocktitle.north east);
}

\useblockstyle{MyBlockStyle}
\useinnerblockstyle{MyInnerBlockStyle}
\newcommand{\myblock}[2]{\block{{\LARGE \bfseries \sffamily \addfontfeature{LetterSpace=3.0} #1}}{\large #2}}


%%--------------------------------------------------
%% TITLE
%%--------------------------------------------------
\definetitlestyle{MyTitle}{
    width=\textwidth,
    roundedcorners=0,
    linewidth=4pt,
    innersep=1.5cm,
    titletotopverticalspace=-1mm,
    titletoblockverticalspace=15mm
}{
    \begin{scope}[line width=\titlelinewidth]
        \fill[
            fill=Foreground
        ] (\titleposleft, \titleposbottom) rectangle (\titleposright, \titlepostop);
        \draw[color=titlebgcolor, line width=10, shorten <=-3pt, shorten >=-3pt]
            (\titleposleft, \titleposbottom) -- (\titleposright, \titleposbottom);
        \path (\titleposleft, \titlepostop) -- ++(0, -1) node [anchor=north west, inner sep=20pt] {\includegraphics[scale=1.2]{figures/unibo_neg.pdf}};
        \path (\titleposright, \titlepostop) -- ++(-0.5, -2) node [anchor=north east, inner sep=20pt] {\includegraphics[scale=0.6]{figures/infn_logo_neg.pdf}};
    \end{scope}
}
\usetitlestyle{MyTitle}

\settitle{
    \centering
    \vbox{
        \vspace*{0.25cm}
        \centering \color{titlefgcolor} {\sffamily \bfseries \Huge \addfontfeature{LetterSpace=5.0} \@title \par}
        \vspace*{0.5cm}
        {\sffamily \Large \@author}\\
        \vspace*{0.25cm}
        {\sffamily {\slshape \@institute}}
    }
}

% vim: nospell
